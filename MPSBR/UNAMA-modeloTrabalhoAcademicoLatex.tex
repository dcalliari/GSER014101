%% abtex2-modelo-trabalho-academico.tex, v-1.9.2 laurocesar
%% Copyright 2012-2014 by abnTeX2 group at http://abntex2.googlecode.com/ 
%%
%% This work may be distributed and/or modified under the
%% conditions of the LaTeX Project Public License, either version 1.3
%% of this license or (at your option) any later version.
%% The latest version of this license is in
%%   http://www.latex-project.org/lppl.txt
%% and version 1.3 or later is part of all distributions of LaTeX
%% version 2005/12/01 or later.
%%
%% This work has the LPPL maintenance status `maintained'.
%% 
%% The Current Maintainer of this work is the abnTeX2 team, led
%% by Lauro César Araujo. Further information are available on 
%% http://abntex2.googlecode.com/
%%
%% This work consists of the files abntex2-modelo-trabalho-academico.tex,
%% abntex2-modelo-include-comandos and abntex2-modelo-references.bib
%%

% ------------------------------------------------------------------------
% ------------------------------------------------------------------------
% abnTeX2: Modelo de Trabalho Academico (tese de doutorado, dissertacao de
% mestrado e trabalhos monograficos em geral) em conformidade com 
% ABNT NBR 14724:2011: Informacao e documentacao - Trabalhos academicos -
% Apresentacao
% ------------------------------------------------------------------------
% ------------------------------------------------------------------------

%-------------------------------------------------------------------------
% Modelo adaptado especificamente para o contexto do PPgSI-EACH-USP por 
% Marcelo Fantinato, com auxílio dos Professores Norton T. Roman, Helton
% H. Bíscaro e Sarajane M. Peres, em 2015, com muitos agradecimentos aos 
% criadores da classe e do modelo base.
%
% 20/06/2017: inclusão de "lista de quadros" com base no especificado em:
% https://github.com/abntex/abntex2/wiki/HowToCriarNovoAmbienteListing,
% de autoria de "Eduardo de Santana Medeiros Alexandre".
%
%-------------------------------------------------------------------------

\documentclass[
	% -- opções da classe memoir --
	12pt,				% tamanho da fonte
	% openright,			% capítulos começam em pág ímpar (insere página vazia caso preciso)
	oneside,			% para impressão apenas no anverso (apenas frente). Oposto a twoside
	a4paper,			% tamanho do papel. 
	% -- opções da classe abntex2 --
	%chapter=TITLE,		% títulos de capítulos convertidos em letras maiúsculas
	%section=TITLE,		% títulos de seções convertidos em letras maiúsculas
	%subsection=TITLE,	% títulos de subseções convertidos em letras maiúsculas
	%subsubsection=TITLE,% títulos de subsubseções convertidos em letras maiúsculas
	% -- opções do pacote babel --
	english,			% idioma adicional para hifenização
	%french,				% idioma adicional para hifenização
	%spanish,			% idioma adicional para hifenização
	brazil				% o último idioma é o principal do documento
	]{abntex2unama}

% ---
% Pacotes básicos 
% ---
\usepackage{lmodern}			% Usa a fonte Latin Modern			
\usepackage[T1]{fontenc}		% Selecao de codigos de fonte.
\usepackage[utf8]{inputenc}		% Codificacao do documento (conversão automática dos acentos)
\usepackage{lastpage}			% Usado pela Ficha catalográfica
\usepackage{indentfirst}		% Indenta o primeiro parágrafo de cada seção.
\usepackage{color}				% Controle das cores
\usepackage{graphicx}			% Inclusão de gráficos
\usepackage{microtype} 			% para melhorias de justificação
\usepackage{pdfpages}     %para incluir pdf
\usepackage{algorithm}			%para ilustrações do tipo algoritmo
\usepackage{mdwlist}			%para itens com espaço padrão da abnt
\usepackage[noend]{algpseudocode}			%para ilustrações do tipo algoritmo
		
% ---
% Pacotes adicionais, usados apenas no âmbito do Modelo Canônico do abnteX2
% ---
\usepackage{lipsum}				% para geração de dummy text
% ---

% ---
% Pacotes de citações
% ---
\usepackage{hyperref}
\usepackage[brazilian,hyperpageref]{backref}	 % Paginas com as citações na bibl
\usepackage[alf,abnt-etal-list=0,abnt-etal-text=it]{abntex2cite}	% Citações padrão ABNT

% --- 
% CONFIGURAÇÕES DE PACOTES
% --- 

% ---
% Configurações do pacote backref
% Usado sem a opção hyperpageref de backref
\renewcommand{\backrefpagesname}{Citado na(s) página(s):~}
% Texto padrão antes do número das páginas
\renewcommand{\backref}{}
% Define os textos da citação
\renewcommand*{\backrefalt}[4]{
	\ifcase #1 %
		Nenhuma citação no texto.%
	\or
		Citado na página #2.%
	\else
		Citado #1 vezes nas páginas #2.%
	\fi}%
% ---

% ---
% Informações de dados para CAPA e FOLHA DE ROSTO
% ---

%-------------------------------------------------------------------------
% Comentário adicional do PPgSI - Informações sobre o ``instituicao'':
%
% Não mexer. Deixar exatamente como está.
%
%-------------------------------------------------------------------------
\instituicao{
	UNIVERSIDADE DA AMAZÔNIA
	\par
	CURSO DE BACHARELADO EM CIÊNCIA DA COMPUTAÇÃO
	\par
	DISCIPLINA DE QUALIDADE  DE SOFTWARE
	}

%-------------------------------------------------------------------------
% Comentário adicional do PPgSI - Informações sobre o ``título'':
%
% Em maiúscula apenas a primeira letra da sentença (do título), exceto 
% nomes próprios, geográficos, institucionais ou Programas ou Projetos ou 
% siglas, os quais podem ter letras em maiúscula também.
%
% O subtítulo do trabalho é opcional.
% Sem ponto final.
%
% Atenção: o título da Dissertação/Tese na versão corrigida não pode mudar. 
% Ele deve ser idêntico ao da versão original.
%
%-------------------------------------------------------------------------
\titulo{Plano para Alcançar o Nível G do MPS.BR: Estratégias e desafios para a implementação do MPS-SW}

%-------------------------------------------------------------------------
% Comentário adicional do PPgSI - Informações sobre o ``autor'':
%
% Todas as letras em maiúsculas.
% Nome completo.
% Sem ponto final.
%-------------------------------------------------------------------------
\autor{\uppercase{Daniel Bahia Pinheiro Calliari\\Maria Luisa Lameirão Sousa\\Marcus Felipe Barradas Andrade}}

%-------------------------------------------------------------------------
% Comentário adicional do PPgSI - Informações sobre o ``local'':
%
% Não incluir o ``estado''.
% Sem ponto final.
%-------------------------------------------------------------------------
\local{Belém}

%-------------------------------------------------------------------------
% Comentário adicional do PPgSI - Informações sobre a ``data'':
%
% Colocar o ano do depósito (ou seja, o ano da entrega) da respectiva 
% versão, seja ela a versão original (para a defesa) seja ela a versão 
% corrigida (depois da aprovação na defesa). 
%
% Atenção: Se a versão original for depositada no final do ano e a versão 
% corrigida for entregue no ano seguinte, o ano precisa ser atualizado no 
% caso da versão corrigida. 
% Cuidado, pois o ano da ``capa externa'' também precisa ser atualizado 
% nesse caso.
%
% Não incluir o dia, nem o mês.
% Sem ponto final.
%-------------------------------------------------------------------------
\data{2024}

%-------------------------------------------------------------------------
% Comentário adicional do PPgSI - Informações sobre o ``Orientador'':
%
% Se for uma professora, trocar por ``Profa. Dra.''
% Nome completo.
% Sem ponto final.
%-------------------------------------------------------------------------
\orientador{Prof. Dr. Fulano de Tal}

%-------------------------------------------------------------------------
% Comentário adicional do PPgSI - Informações sobre o ``Coorientador'':
%
% Opcional. Incluir apenas se houver co-orientador formal, de acordo com o 
% Regulamento do Programa.
%
% Se for uma professora, trocar por ``Profa. Dra.''
% Nome completo.
% Sem ponto final.
%-------------------------------------------------------------------------
\coorientador{Prof. Dr. Fulano de Tal}

\tipotrabalho{Dissertação (Mestrado) / Tese (Doutorado)}

\preambulo{
%-------------------------------------------------------------------------
% Comentário adicional do PPgSI - Informações sobre o texto ``Versão 
% original'':
%
% Não usar para Qualificação.
% Não usar para versão corrigida de Dissertação/Tese.
%
%-------------------------------------------------------------------------
Versão original \newline \newline \newline 
%-------------------------------------------------------------------------
% Comentário adicional do PPgSI - Informações sobre o ``texto principal do
% preambulo'':
%
% Para Doutorado, trocar por: Tese apresentada à Escola de Artes, Ciências e Humanidades da Universidade de São Paulo para obtenção do título de Doutor (ou Doutora) em Ciências pelo Programa de Pós-graduação em Sistemas de Informação. 
%
% Para Qualificação, trocar por: Projeto de pesquisa para exame de qualificação apresentado à Escola de Artes, Ciências e Humanidades da Universidade de São Paulo como parte dos requisitos para obtenção do título de Mestre (ou Doutor ou Doutora) em Ciências pelo Programa de Pós-graduação em Sistemas de Informação.
%
%-------------------------------------------------------------------------
Trabalho acadêmico apresentado à Universidade da Amazônia como parte de disciplinas do Curso de Bacharelado em Ciência da Computação.
%
\newline \newline Área de concentração: Qualidade de Software
%-------------------------------------------------------------------------
% Comentário adicional do PPgSI - Informações sobre o texto da ``Versão 
% corrigida'':
%
% Não usar para Qualificação.
% Não usar para versão original de Dissertação/Tese.
% 
% Substituir ``xx de xxxxxxxxxxxxxxx de xxxx'' pela ``data da defesa''.
%
%-------------------------------------------------------------------------
% \newline \newline \newline Versão corrigida contendo as alterações solicitadas pela comissão julgadora em xx de xxxxxxxxxxxxxxx de xxxx. A versão original encontra-se em acervo reservado na Biblioteca da EACH-USP e na Biblioteca Digital de Teses e Dissertações da USP (BDTD), de acordo com a Resolução CoPGr 6018, de 13 de outubro de 2011.
}
% ---


% ---
% Configurações de aparência do PDF final

% alterando o aspecto da cor azul
\definecolor{blue}{RGB}{41,5,195}

% informações do PDF
\makeatletter
\hypersetup{
     	%pagebackref=true,
		pdftitle={\@title}, 
		pdfauthor={\@author},
    	pdfsubject={\imprimirpreambulo},
	    pdfcreator={laTeX com abnTeX2 adaptado para a UNAMA/ALC},
		pdfkeywords={abnt}{latex}{abntex}{abntex2unama}{trabalho acadêmico}{unama}, 
		colorlinks=true,       		% false: boxed links; true: colored links
    	linkcolor=blue,          	% color of internal links
    	citecolor=blue,        		% color of links to bibliography
    	filecolor=magenta,      		% color of file links
		urlcolor=blue,
		bookmarksdepth=4
}
\makeatother
% --- 

% --- 
% Espaçamentos entre linhas e parágrafos 
% --- 

% O tamanho do parágrafo é dado por:
\setlength{\parindent}{1.25cm}

% Controle do espaçamento entre um parágrafo e outro:
\setlength{\parskip}{0cm}  % tente também \onelineskip
\renewcommand{\baselinestretch}{1.5}

% ---
% compila o indice
% ---
\makeindex
% ---

	% Controlar linhas orfas e viuvas
  \clubpenalty10000
  \widowpenalty10000
  \displaywidowpenalty10000

% ----
% Início do documento
% ----
\begin{document}

% Retira espaço extra obsoleto entre as frases.
\frenchspacing

% ----------------------------------------------------------
% ELEMENTOS PRÉ-TEXTUAIS
% ----------------------------------------------------------
% \pretextual

% ---
% Capa
% ---
%-------------------------------------------------------------------------
% Comentário adicional do PPgSI - Informações sobre a ``capa'':
%
% Esta é a ``capa'' principal/oficial do trabalho, a ser impressa apenas 
% para os casos de encadernação simples (ou seja, em ``espiral'' com 
% plástico na frente).
% 
% Não imprimir esta ``capa'' quando houver ``capa dura'' ou ``capa brochura'' 
% em que estas mesmas informações já estão presentes nela.
%
%-------------------------------------------------------------------------
\imprimircapa
% ---

% ---
% inserir o sumario
% ---
\pdfbookmark[0]{\contentsname}{toc}
\tableofcontents*
\cleardoublepage
% ---



% ----------------------------------------------------------
% ELEMENTOS TEXTUAIS
% ----------------------------------------------------------
\textual



%-------------------------------------------------------------------------
% Comentário adicional do PPgSI - Informações sobre ``títulos de seções''
% 
% Para todos os títulos (seções, subseções, tabelas, ilustrações, etc.):
%
% Em maiúscula apenas a primeira letra da sentença (do título), exceto 
% nomes próprios, geográficos, institucionais ou Programas ou Projetos ou
% siglas, os quais podem ter letras em maiúscula também.
%
%-------------------------------------------------------------------------
\chapter{Introdução}
O Modelo de Melhoria de Processo do Software Brasileiro (MPS.Br) é uma iniciativa que visa aprimorar os processos de desenvolvimento de software nas empresas \cite{mpsbr}. Ele foi desenvolvido para atender à realidade das organizações brasileiras, oferecendo uma alternativa acessível e eficaz para alcançar a excelência em processos de software. Composto por níveis de maturidade, o modelo fornece um guia progressivo para melhorar a qualidade dos processos e produtos de software. Cada nível apresenta requisitos específicos, permitindo que as empresas melhorem gradualmente seus processos de forma sustentável.

O MPS.Br é particularmente relevante para empresas que buscam maior competitividade no mercado, pois a qualidade dos processos de software tem um impacto direto na entrega de produtos eficientes e confiáveis \cite{mpsbr_benefits}. A implementação desse modelo exige comprometimento organizacional, pois implica em mudanças culturais e estruturais. Apesar disso, os benefícios superam os desafios, como a padronização de processos, aumento de produtividade e satisfação dos clientes.

Neste contexto, este trabalho propõe um plano para alcançar o Nível G do MPS.Br, que representa um estágio inicial de maturidade nos processos de software. O objetivo é apresentar estratégias e desafios para a implementação do MPS-SW, focando nos processos de Gerência de Projetos e Gerência de Requisitos. Esses processos são essenciais para estabelecer uma base sólida de controle e previsibilidade no desenvolvimento de software, contribuindo para a melhoria contínua da qualidade e eficiência organizacional.

\chapter{Contexto do MPS.Br}
O MPS.Br é um modelo de referência para a melhoria de processos de software, baseado nas normas ISO/IEC 12207 e ISO/IEC 15504. Ele foi desenvolvido por um consórcio de instituições brasileiras, com o apoio do governo e da indústria de software. O modelo é composto por sete níveis de maturidade, que representam estágios evolutivos nos processos de software. Cada nível possui requisitos específicos, que devem ser atendidos para alcançar a maturidade desejada. \cite{iso12207, iso15504}

O MPS.Br é uma alternativa flexível e adaptável, que pode ser aplicada a organizações de diferentes portes e segmentos. Ele oferece um roteiro claro para a melhoria contínua dos processos, permitindo que as empresas identifiquem suas necessidades e prioridades. O modelo também fornece um conjunto de práticas e diretrizes, que orientam as organizações na implementação de processos eficazes e eficientes \cite{mpsbr_guide}.

A adoção do MPS.Br traz diversos benefícios para as organizações, como a redução de custos, o aumento da produtividade e a melhoria da qualidade dos produtos de software \cite{mpsbr_benefits}. Além disso, o modelo contribui para a padronização e a documentação dos processos, facilitando a comunicação e o compartilhamento de conhecimento. Com essas vantagens, o MPS.Br se tornou uma referência importante para as empresas que buscam a excelência em processos de software.

\chapter{Objetivo do Nível G}
O Nível G do MPS.Br tem como principal objetivo alcançar processos parcialmente gerenciados. Esse estágio inicial de maturidade foca na organização e no controle básico dos processos de software. Os dois principais processos tratados neste nível são a Gerência de Projetos e a Gerência de Requisitos \cite{mpsbr_levelG}. A Gerência de Projetos busca estabelecer práticas consistentes para planejar, executar e monitorar projetos, garantindo o cumprimento de prazos e metas estabelecidas. Já a Gerência de Requisitos concentra-se em identificar, documentar e gerenciar os requisitos do software, assegurando que eles sejam compreendidos e atendidos.

Esses processos são fundamentais para estabelecer uma base sólida de controle e previsibilidade no desenvolvimento de software. O Nível G não requer a total maturidade de todos os processos, mas incentiva que eles sejam documentados, aplicados de forma consistente e sujeitos a melhorias contínuas. Essa abordagem inicial é essencial para criar um ambiente propício à implementação dos níveis mais avançados do MPS.Br.

\chapter{Requisitos do Nível G}

O Nível G do MPS.Br possui requisitos específicos para a Gerência de Projetos e a Gerência de Requisitos. Para a Gerência de Projetos, os principais requisitos são \cite{mpsbr_requirements}:

\begin{itemize}
	\item Estabelecer um Plano de Gerência de Projetos (PGP), que defina o escopo, os recursos, os riscos e o cronograma do projeto.
	\item Realizar reuniões de acompanhamento periódicas, para monitorar o progresso do projeto e identificar desvios em relação ao planejado.
	\item Estabelecer métricas de desempenho, para avaliar a qualidade e a eficiência dos processos de gerência de projetos.
	\item Documentar lições aprendidas, para registrar as experiências e os resultados obtidos ao longo do projeto.
	\item Realizar avaliações pós-projeto, para identificar pontos fortes e áreas de melhoria nos processos de gerência de projetos.
	\item Estabelecer um Plano de Gerência de Requisitos (PGR), que defina a identificação, análise e rastreabilidade dos requisitos do software.
	\item Realizar revisões de requisitos, para validar a compreensão e a adequação dos requisitos do software.
	\item Estabelecer um processo de controle de mudanças, para gerenciar as alterações nos requisitos do software ao longo do ciclo de vida do projeto.
\end{itemize}

Esses requisitos são essenciais para garantir que os processos de Gerência de Projetos e Gerência de Requisitos sejam aplicados de forma consistente e eficaz. Eles fornecem uma base sólida para a implementação do Nível G do MPS.Br, contribuindo para a melhoria contínua dos processos de software.

\chapter{Desafios para Implementação}
A implementação do MPS.Br, especialmente no Nível G, enfrenta diversos desafios. Um dos mais frequentes é a resistência à mudança. Funcionários e gestores podem se sentir desconfortáveis ao abandonar métodos e práticas consolidados, mesmo que sejam ineficientes \cite{change_management}. Essa resistência pode ser minimizada com uma comunicação eficaz, que demonstre os benefícios da mudança e envolva todos os níveis da organização no processo de transformação.

Outro desafio importante é a falta de conhecimento sobre o modelo. Muitas organizações precisam capacitar suas equipes para compreender os requisitos do MPS.Br e sua aplicabilidade no dia a dia. A integração dos novos processos aos já existentes também pode ser complexa, pois requer um alinhamento cuidadoso para evitar conflitos. Além disso, a alocação de recursos, tanto financeiros quanto humanos, é um ponto crítico. A falta de planejamento adequado pode gerar atrasos e aumento de custos, impactando negativamente o cronograma da implementação \cite{resource_allocation}.

Por fim, a falta de comprometimento da alta direção é um desafio significativo. Sem o apoio e o engajamento dos líderes da organização, a implementação do MPS.Br pode ser vista como uma iniciativa isolada, sem impacto real nos processos de software. É fundamental que a alta direção esteja envolvida desde o início, demonstrando seu compromisso com a melhoria contínua e a excelência operacional \cite{leadership_commitment}.

\chapter{Estratégias de Implementação}
Para superar os desafios da implementação do Nível G, é essencial adotar estratégias bem definidas. O treinamento e a capacitação da equipe são passos fundamentais. Workshops e cursos específicos sobre o MPS-SW ajudam a disseminar o conhecimento necessário, enquanto treinamentos em Gerência de Projetos e Requisitos garantem que os processos essenciais sejam compreendidos e aplicados de forma eficaz \cite{training}.

Outra estratégia é engajar a equipe em todos os níveis hierárquicos. A comunicação clara dos benefícios do MPS.Br, como a melhoria na qualidade e na eficiência dos processos, ajuda a conquistar o apoio e o comprometimento dos colaboradores. O planejamento e o monitoramento contínuo também são cruciais. Estabelecer metas claras, prazos realistas e realizar avaliações periódicas do progresso garantem que o projeto de implementação esteja alinhado aos objetivos organizacionais \cite{strategic_planning}.

O uso de ferramentas e métodos adequados é outra estratégia importante. Softwares de gerenciamento de projetos, como o Microsoft Project e o Trello, facilitam o planejamento e o acompanhamento das atividades. Métodos ágeis, como Scrum e Kanban, podem ser adotados para promover a colaboração e a flexibilidade nos processos de desenvolvimento de software \cite{agile_methods}. A combinação dessas estratégias contribui para o sucesso da implementação do Nível G do MPS.Br.

\chapter{Documentação Necessária}
A documentação desempenha um papel crítico na implementação do Nível G do MPS.Br. O Plano de Gerência de Projetos (PGP) é um dos documentos essenciais, pois detalha o escopo, os recursos, os riscos e o cronograma do projeto. Ele serve como guia para todas as atividades de gerenciamento, garantindo que o projeto seja conduzido de maneira organizada e eficiente \cite{pgp}.

Outro documento fundamental é o Plano de Gerência de Requisitos (PGR). Este plano abrange a identificação, análise e rastreabilidade dos requisitos do projeto, assegurando que as necessidades dos stakeholders sejam compreendidas e atendidas. Relatórios de acompanhamento, que fornecem métricas de desempenho e status dos projetos, também são indispensáveis para monitorar o progresso. Além disso, políticas internas e documentação de treinamento ajudam a padronizar os processos e a alinhar as expectativas da equipe \cite{pgr}.

A documentação deve ser clara, concisa e acessível a todos os envolvidos no projeto. Ela deve ser revisada e atualizada regularmente, para refletir as mudanças e melhorias nos processos. A documentação adequada é um dos pilares da implementação bem-sucedida do MPS.Br, garantindo que os processos sejam compreendidos, aplicados e aprimorados de forma consistente.

\chapter{Conclusão}
A implementação do Nível G do MPS.Br é um processo desafiador, mas viável, desde que seja abordado com planejamento estratégico e comprometimento. O treinamento adequado, o engajamento da equipe e o uso de ferramentas e métodos apropriados são essenciais para superar os obstáculos comuns. A integração de novos processos com os já existentes deve ser feita de forma cuidadosa, garantindo harmonia e eficiência operacional.

Os benefícios da implementação incluem maior controle sobre os projetos, melhoria na qualidade dos produtos e otimização dos processos organizacionais. Embora os desafios sejam significativos, os ganhos em competitividade e satisfação do cliente tornam o esforço valioso. O sucesso nessa jornada exige um compromisso contínuo com a melhoria e a adaptação, estabelecendo uma base sólida para alcançar níveis mais avançados do MPS.Br \cite{mpsbr_conclusion}.

% ----------------------------------------------------------
% ELEMENTOS PÓS-TEXTUAIS
% ----------------------------------------------------------
\postextual
% ----------------------------------------------------------

% ----------------------------------------------------------
% Referências bibliográficas
% ----------------------------------------------------------
\bibliography{referencias}

% ----------------------------------------------------------
% Glossário
% ----------------------------------------------------------
%
% Consulte o manual da classe abntex2 para orientações sobre o glossário.
%
%\glossary

%---------------------------------------------------------------------
% INDICE REMISSIVO
%---------------------------------------------------------------------
%%%%%MF\phantompart
%%%%%MF\printindex
%---------------------------------------------------------------------

\end{document}
